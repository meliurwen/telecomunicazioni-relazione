\onehalfspacing
OpenFlow fornisce tutti i mezzi principali per configurare switch compatibili ed applicare le principali politiche di sicurezza. Come già accennato nel capitolo \ref{works}, gli articoli diversi da quello di \textit{Bakker} che l'autore di questo testo ha consultato sembrano non sfruttare appieno la flessibilità offerta da questa tecnologia; consentendo solo di configurare un solo switch o di farlo singolarmente per ognuno.

La soluzione proposta dal gruppo di \textit{Bakker} presenta un meccanismo che combina questi due approcci in modo che gli host all'interno di una rete possano essere protetti dalle minacce \textbf{\textit{sia interne che esterne}}, facilitando al contempo il settaggio di diverse operazioni di filtraggio tra gruppi di switch diversi.

Oltretutto, \textbf{\textit{come nota personale}}, l'autore di questo testo si sente di aggiungere la concatenzione nella pipeline (cap. \ref{pipeline}) di una tabella con politiche permissive succeduta da una con politiche restrittive permette di semplificare sia la logica che il numero di regole necessarie; questo, oltre a conferire un ovvio vantaggio sulla mantenibilità da parte degli amministratori di rete, permette anche di garantire prestazioni ottimali, visto che come mostra il lavoro del gruppo di \textit{Collings} \cite{collings2014openflow}, sono molto legate (specialmente per quanto riguarda la latenza) al numero di regole applicate.
